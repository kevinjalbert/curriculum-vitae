%%
% Author: Kevin Jalbert
% Email: kevin.j.jalbert@gmail.com
% Website: http://kevinjalbert.com
% Repository: https://github.com/kevinjalbert/cv_kevinjalbert
% Version: 1.2.0
% Info: This CV is built in such a manner that it becomes quite easy to add
%       new sections and entries to each section. There are PDF bookmarks for
%       each section as well.
%

%%%%%%%%%%%%%%
%% PACKAGES %%
%%%%%%%%%%%%%%
% The set of required packages to compile this LaTeX document.
%
\documentclass[letterpaper,10pt,oneside]{article}
\usepackage{ifthen}
\usepackage{ulem}
\usepackage[left=1in,
            right=1in,
            top=1in,
            bottom=1in,
            nohead,
            nofoot]{geometry}
\usepackage[bookmarks,
            pdftitle={Curriculum Vitae},
            pdfauthor={Kevin Jalbert},
            pdfsubject={},
            pdfcreator={},
            pdfproducer={},
            pdfkeywords={},
            pdfpagemode={},
            pdfstartview=FitH,
            colorlinks=true,
            linkcolor=black,
            anchorcolor=black,
            citecolor=black,
            filecolor=black,
            menucolor=black,
            urlcolor=black]{hyperref}

%%%%%%%%%%%%%%%%%%%%%%
%% PERSONAL DETAILS %%
%%%%%%%%%%%%%%%%%%%%%%
% The personal information of the person who is using this resume template.
% It is possible to include \href links in these commands.
%

% putName(newcommand)
% Name of the user for this CV.
% [1]: null
%
\newcommand{\putName}[1]{
  Kevin Jalbert
}

% putWebsite(newcommand)
% Link to the user's website, could be a href link.
% [1]: null
%
\newcommand{\putWebsite}[1]{
  \href{http://kevinjalbert.com/}{kevinjalbert.com}
}

% putAddress(newcommand)
% The user's city address.
% [1]: null
%
\newcommand{\putAddress}[1]{
  292 Orange Cres, L1G 5X3
}

% putRegion(newcommand)
% The user's regional address.
% [1]: null
%
\newcommand{\putRegion}[1]{
  Oshawa, Ontario, Canada
}

% putEmail(newcommand)
% The user's email address, could be a href link.
% [1]: null
%
\newcommand{\putEmail}[1]{
  \href{mailto:kevin.j.jalbert@gmail.com}{kevin.j.jalbert@gmail.com}
}

% putPhone(newcommand)
% The user's phone number.
% [1]: null
%
\newcommand{\putPhone}[1]{
  +1 (905) 924-5030
}

%%%%%%%%%%%%%%%%%%
%% MACROS/STYLE %%
%%%%%%%%%%%%%%%%%%
% To ease the insertion of information into the resume a set of function-like
% commands exist. The information is passed into the command which is then
% formated as desired. In addition there exist other helper commands.
%

% Force a clean looking page
\pagestyle{empty}
\setlength{\parindent}{0cm}

% indenter(newenvironment)
% Statement that returns the parindent back to 0cm, used with blockIndent.
% [1]: null
%
\newenvironment{indenter}[1]{
  \setlength{\parindent}{0cm}
}

% blockIndent(newenvironment)
% Statements within this environment will be indented 1cm.
% [1]: null
%
\newenvironment{blockIndent}[1]{
  \begin{indenter}{}
    \setlength{\parindent}{0.25cm}
    \hangindent=0.25cm
}{\end{indenter}}

% putContact(newcommand)
% Displays the contact information for the user (address, region, email and
% phone) that have been acquire in the personal details section.
% [1]: null
%
\newcommand{\putContact}[1]{
  \begin{blockIndent}{}
    \putAddress{} \hfill \textit{\putEmail{}} \\
    \putRegion{} \hfill \putPhone{} \\
  \end{blockIndent}
}

% putTitle(newcommand)
% Displays the title of the CV (name, website, and contact information). The
% Title is also bookmarked for quick access.
% [1]: null
%
\newcommand{\putTitle}[1]{
  \normalsize
  \pdfbookmark{Curriculum Vitae}{Curriculum Vitae}{
    \hspace{-0.5cm} \textsc{\Huge{\putName{}} \hfill \LARGE{\putWebsite{}}}\\
  }
  \putContact{}
  \hrule
}

% section(renewcommand)
% Displays the section's title, as well as placing a bookmark.
% [1]: Section's title
%
\renewcommand{\section}[1]{
  \vspace{3mm}
  \pdfbookmark{#1}{#1}{
    \textsc{\Large{\underline{#1}}}
    \vspace{3mm}
  }
}

% generalEntry(newcommand)
% Displays a general entry (a simple paragraph, or anything really)
% [1]: Contents of the general entry
%
\newcommand{\generalEntry}[1]{
  \begin{blockIndent}{}
    #1 \\
  \end{blockIndent}
}

% educationEntry(newcommand)
% Displays an entry about the user's educational background.
% [1]: University/college's Name
% [2]: Region
% [3]: Degree/diploma
% [4]: Start date
% [5]: End date
% [6]: Note/comment (optional)
%
\newcommand{\educationEntry}[6]{
  \begin{blockIndent}{}
    \begin{minipage}[t]{0.645\linewidth}
      \raggedright
      #1 \\
      \textbf{#3} \\
      \ifthenelse{
        \equal{#6} {}}
        {} {#6 \\}
    \end{minipage}
    \begin{minipage}[t]{0.335\linewidth}
      \raggedleft
      #2 \\
      #4 -- #5 \\
    \end{minipage}
  \end{blockIndent}
  \vspace{0.5cm}
}

% researchEntry(newcommand)
% Displays an entry about the user's research experiences.
% [1]: Research group/affiliation
% [2]: University/college's name
% [3]: Region
% [4]: Position's title
% [5]: Start date
% [6]: End date
% [7]: Colleagues (follow this format "title: name")
% [8]: Research title
% [9]: Research description
%
\newcommand{\researchEntry}[9]{
  \begin{blockIndent}{}
    \begin{minipage}[t]{0.645\linewidth}
      \raggedright
      \textbf{#1}, #2 \\
      #4 \\
      #7 \\
      \textit{#8:} #9
    \end{minipage}
    \begin{minipage}[t]{0.335\linewidth}
      \raggedleft
      #3 \\
      #5 -- #6
    \end{minipage}
  \end{blockIndent}
  \vspace{0.5cm}
}

% teachingEntry(newcommand)
% Displays an entry about the user's teaching experiences.
% [1]: Position's title
% [2]: University/College's name
% [3]: Region
% [4]: Course code
% [5]: Start date
% [6]: End date
% [7]: Course title
% [8]: Course description (optional)
%
\newcommand{\teachingEntry}[8]{
  \begin{blockIndent}{}
    \begin{minipage}[t]{0.645\linewidth}
      \raggedright
      \textbf{#1}, #2 \\
      Course Code: #4 \\
      \textit{#7} \\
      \ifthenelse{
        \equal{#8} {}}
        {} {Responsibilities: #8 \\}
    \end{minipage}
    \begin{minipage}[t]{0.335\linewidth}
      \raggedleft
      #3 \\
      #5 -- #6 \\
    \end{minipage}
  \end{blockIndent}
  \vspace{0.5cm}
}

% academicAwardsEntry(newcommand)
% Displays an entry about the user's academic awards and honors.
% [1]: Award/honor title
% [2]: Institution's name where it was received
% [3]: Start date
% [4]: End date
% [5]: Notes/description (optional)
%
\newcommand{\academicAwardsEntry}[5]{
  \begin{blockIndent}{}
    \begin{minipage}[t]{0.645\linewidth}
      \raggedright
      \textbf{#1}, #2 \\
      \ifthenelse{
        \equal{#5} {}}
        {} {#5 \\}
    \end{minipage}
    \begin{minipage}[t]{0.335\linewidth}
    \raggedleft
      #3 -- #4
    \end{minipage}
  \end{blockIndent}
  \vspace{0.5cm}
}

% publicationEntry(newcommand)
% Displays an entry about the user's publications.
% [1]: Authors
% [2]: Publication title
% [3]: Proceedings
% [4]: Location in Proceedings (ex: volume 4, pages 3 --10)
% [5]: Region
% [6]: Date
% [7]: PDF/website url (optional)
%
\newcommand{\publicationEntry}[7]{
  \begin{blockIndent}{}
    #1.
    \ifthenelse{
      \equal{#7} {}}
      {\uline{``#2''}} {\href{#7}{\uline{``#2''}}},
    \textit{#3}, #4, #5, #6. \\
  \end{blockIndent}
}

% posterEntry(newcommand)
% Displays an entry about the user's posters.
% [1]: Authors
% [2]: Poster/exhibit title
% [3]: Proceedings
% [4]: Region
% [5]: Date
% [6]: PDF/website url (optional)
%
\newcommand{\posterEntry}[6]{
  \begin{blockIndent}{}
    #1.
    \ifthenelse{
      \equal{#6} {}}
      {\uline{``#2''}} {\href{#6}{\uline{``#2''}}},
    \textit{#3}, #4, #5. \\
  \end{blockIndent}
}

% presentationkEntry(newcommand)
% Displays an entry about the user's given presentations.
% [1]: Presentation title
% [2]: Event
% [3]: Region
% [4]: Date
% [5]: Topic
% [6]: PDF/website url (optional)
%
\newcommand{\presentationEntry}[6]{
  \begin{blockIndent}{}
    \begin{minipage}[t]{0.645\linewidth}
      \raggedright
      \ifthenelse{
        \equal{#6} {}}
        {\textsc{#1}} {\href{#6}{\textsc{#1}}} \\
      Event: #2 \\
      Summary: #5
    \end{minipage}
    \begin{minipage}[t]{0.335\linewidth}
      \raggedleft
      #3 \\
      #4
    \end{minipage}
  \end{blockIndent}
  \vspace{0.5cm}
}

% organizationInvolvementEntry(newcommand)
% Displays an entry about the user's involvement in an organization.
% [1]: Organization title
% [2]: Position title
% [3]: Start date
% [4]: End date
% [5]: Website url (optional)
%
\newcommand{\organizationInvolvementEntry}[5]{
  \begin{blockIndent}{}
    \begin{minipage}[t]{0.645\linewidth}
      \raggedright
      \ifthenelse{
        \equal{#5} {}}
        {\textsc{#1}} {\href{#5}{\textsc{#1}}} \\
      Role: #2
    \end{minipage}
    \begin{minipage}[t]{0.335\linewidth}
      \raggedleft
      #3 -- #4
    \end{minipage}
  \end{blockIndent}
  \vspace{0.5cm}
}

% skillEntry(newcommand)
% Displays an entry about the user's skill. The skill level represents the
% user's level of comfort level of expertise is graphed out of a 10 scale.
% [1]: Skill
% [2]: Level of comfort out of 10 (whole numbers)
%
\newcommand{\skillEntry}[2]{
  \begin{blockIndent}{}
    \begin{tabular}{p{5cm} l}
      \small{#1} & \rule{#2em}{1.75mm}\rule{#2em}{1.75mm} \tiny{#20\%}
    \end{tabular}
  \end{blockIndent}
}

% hide(newcommand)
% Hides the contained statements (useful for tailoring a CV).
% [1]: Statements to hide
%
\newcommand{\hide}[1]{}

%%%%%%%%%%%%%%%%%%%%%%
%% CURRICULUM VITAE %%
%%%%%%%%%%%%%%%%%%%%%%
% Begin the CV document. The actual content for the CV goes here.
%
\begin{document}

  % Place the title for the CV
  \putTitle{}

  %%%%%%%%%%%%%%%
  %% OBJECTIVE %%
  %%%%%%%%%%%%%%%
  % The user's objective is displayed here.
  %
  \section{Objective}

  \generalEntry{
    To acquire a position that utilizes my strong software engineering background and computer science skill set. My ideal position would involve work that is challenging and exciting, while retaining a relaxed atmosphere. The ability to contribute to open-source software is enticing and beneficial to personal/professional growth.
  }

  %%%%%%%%%%%%%%%
  %% EDUCATION %%
  %%%%%%%%%%%%%%%
  % The user's educational background is displayed here.
  %
  \section{Education}

  \educationEntry
    {\href{http://www.uoit.ca/}{\textbf{University of Ontario Institute of Technology (UOIT)}}}
    {Oshawa, Ontario, Canada}
    {Masters of Science in \href{http://gradstudies.uoit.ca/EN/main/future\_students/masters\_programs/computerscience.html}{Computer Science}}
    {Sep. 2010}
    {Present}
    {Cumulative GPA of 4.15 / 4.30.}

  \educationEntry
    {\href{http://www.uoit.ca/}{\textbf{UOIT}}}
    {Oshawa, Ontario, Canada}
    {Bachelor of \href{http://engineering.uoit.ca/undergraduate/software/}{Software Engineering}}
    {Sep. 2006}
    {Apr. 2010}
    {Cumulative GPA of 3.44 / 4.30.}

  %%%%%%%%%%%%%%%%%%%%%%%%%
  %% RESEARCH EXPERIENCE %%
  %%%%%%%%%%%%%%%%%%%%%%%%%
  % The user's research experience is displayed here.
  %
  \section{Research Experience}

  \researchEntry
    {\href{http://sqrg.ca/}{\textbf{Software Quality Research Group}}}
    {\href{http://www.uoit.ca/}{UOIT}}
    {Oshawa, Ontario, Canada}
    {Graduate Research Assistant}
    {Aug. 2010}
    {Present}
    {Supervisor: Dr. Jeremy S. Bradbury}
    {Software Quality and Testing}
    {Working on research projects related to Concurrency, Model Checking and Software Metrics.}

  \researchEntry
    {\href{http://sqrg.ca/}{\textbf{Software Quality Research Group}}}
    {\href{http://www.uoit.ca/}{UOIT}}
    {Oshawa, Ontario, Canada}
    {Undergraduate Research Assistant}
    {May 2010}
    {Aug. 2010}
    {Supervisor: Dr. Jeremy S. Bradbury}
    {A Tool for Automatically Repairing Concurrency Bugs}
    {This research project focuses on the development of a new approach to automatically correct Java concurrency bugs using genetic algorithms.}

  \researchEntry
    {\href{http://sqrg.ca/}{\textbf{Software Quality Research Group}}}
    {\href{http://www.uoit.ca/}{UOIT}}
    {Oshawa, Ontario, Canada}
    {Undergraduate Research Assistant}
    {May 2009}
    {Aug. 2009}
    {Supervisor: Dr. Jeremy S. Bradbury}
    {Using Clone Detection to Identify Bugs in Concurrent Software}
    {This research project focuses on the development of a new approach for regression testing concurrent Java programs using clone detection of bug patterns.}

  %%%%%%%%%%%%%%%%%%%%%%%%%
  %% TEACHING EXPERIENCE %%
  %%%%%%%%%%%%%%%%%%%%%%%%%
  % The user's research experience is displayed here.
  %
  \section{Teaching Experience}

  \teachingEntry
    {Teaching Assistant}
    {\href{http://www.uoit.ca/}{UOIT}}
    {Oshawa, Ontario, Canada}
    {CSCI 1030U \textit{(Two Section)}}
    {Sep. 2011}
    {Dec. 2011}
    {Introduction to Computer Science}
    {Introduced the basic concepts of computer science as well as providing an introduction to computer programming.}

  \teachingEntry
    {Teaching Assistant}
    {\href{http://www.uoit.ca/}{UOIT}}
    {Oshawa, Ontario, Canada}
    {CSCI 2030U \textit{(Two Section)}}
    {Jan. 2011}
    {Apr. 2011}
    {Programming Workshop}
    {Introduced modern concepts in program design and construction along with advanced features of modern object oriented programming languages.}

  \teachingEntry
    {Teaching Assistant}
    {\href{http://www.uoit.ca/}{UOIT}}
    {Oshawa, Ontario, Canada}
    {CSCI 2010U \textit{(One Section)}}
    {Sep. 2010}
    {Dec. 2010}
    {Principles of Computer Science}
    {Introduced students to general computer programming principles and the analysis of algorithms and data structures.}

  \teachingEntry
    {Teaching Assistant}
    {\href{http://www.uoit.ca/}{UOIT}}
    {Oshawa, Ontario, Canada}
    {CSCI 3040U \textit{(One Section)}}
    {Sep. 2010}
    {Dec. 2010}
    {Software Engineering I: Requirements, Design and Analysis}
    {Introduced students to the development of software systems including systems that consist of multiple programs with long life cycles.}

  %%%%%%%%%%%%%%%%%%%%%%%%%%%%%%
  %% ACADEMIC HONORS & AWARDS %%
  %%%%%%%%%%%%%%%%%%%%%%%%%%%%%%
  % The user's academic honors and awards are displayed here.
  %
  \section{Academic Honors and Awards}{}

  \academicAwardsEntry
    {\href{https://osap.gov.on.ca/OSAPPortal/en/A-ZListofAid/TCONT003465.html}{OGS - External Competition}}
    {\href{http://uoit.ca/}{UOIT}}
    {Sep. 2011}
    {Sep. 2012}
    {Scholarship for graduate studies in Ontario. Only 3000 recipients.}

  \academicAwardsEntry
    {Dean's Graduate Scholarship -- Master’s Level}
    {\href{http://uoit.ca/}{UOIT}}
    {Sep. 2011}
    {Sep. 2012}
    {Scholarship for graduate studies at UOIT. Recipients must have at least an 3.7 GPA.}

  \academicAwardsEntry
    {\href{http://gradstudies.uoit.ca/test/EN/main/future\_students/awards\_and\_funding/external\_awards/OGSInstitutional.html}{OGS - Institutional Competition}}
    {\href{http://uoit.ca/}{UOIT}}
    {Sep. 2010}
    {Sep. 2011}
    {Scholarship for graduate studies at UOIT. Only 10 recipients.}

  \academicAwardsEntry
    {Dean's Graduate Scholarship -- Master’s Level}
    {\href{http://uoit.ca/}{UOIT}}
    {Sep. 2010}
    {Sep. 2011}
    {Scholarship for graduate studies at UOIT. Recipients must have at least an 3.7 GPA.}

  \academicAwardsEntry
    {\href{http://www.nserc-crsng.gc.ca/students-etudiants/ug-pc/usra-brpc_eng.asp}{NSERC -- Undergraduate Student Research Award}}
    {\href{http://uoit.ca/}{UOIT}}
    {May 2010}
    {Aug. 2010}
    {Funding for undergraduate summer research at UOIT. Only 14 recipients at UOIT.}

  \academicAwardsEntry
    {\href{http://www.nserc-crsng.gc.ca/students-etudiants/ug-pc/usra-brpc_eng.asp}{NSERC -- Undergraduate Student Research Award}}
    {\href{http://uoit.ca/}{UOIT}}
    {May 2009}
    {Aug. 2009}
    {Funding for undergraduate summer research at UOIT. Only 14 recipients at UOIT.}

  \academicAwardsEntry
    {President's List}
    {\href{http://uoit.ca/}{UOIT}}
    {Sep. 2008}
    {Apr. 2011}
    {For attaining a semester GPA of at least 3.80/4.30 during a undergraduate degree.}

  %%%%%%%%%%%%%%%%%%
  %% PUBLICATIONS %%
  %%%%%%%%%%%%%%%%%%
  % The user's publications are displayed here.
  %
  \section{Publications}

  \publicationEntry
    {Jeremy S. Bradbury and \textbf{Kevin Jalbert}}
    {Automatic Repair of Concurrency Bugs}
    {In Proc. of the 2nd International Symposium on Search Based Software Engineering (SSBSE 2010) - Fast Abstracts}
    {pages 2}
    {Benevento, Italy}
    {Sep. 2010}
    {http://kevinjalbert.com/public/files/publications/SSBSE10.pdf}

  \publicationEntry
    {\textbf{Kevin Jalbert} and Jeremy S. Bradbury}
    {Using Clone Detection to Identify Bugs in Concurrent Software}
    {In Proc. of 26th IEEE International Conference on Software Maintenance (ICSM 2010)}
    {pages 5}
    {Timisoara, Romania}
    {Sep. 2010}
    {http://kevinjalbert.com/public/files/publications/ICSM10.pdf}

  \publicationEntry
    {Jeremy S. Bradbury and \textbf{Kevin Jalbert}}
    {Defining a Catalog of Programming Anti-Patterns for Concurrent Java}
    {In Proc. of the 3rd International Workshop on Software Patterns and Quality (SPAQu'09)}
    {pages 6 -- 11}
    {Orlando, Florida, USA}
    {Nov. 2009}
    {http://kevinjalbert.com/public/files/publications/SPAQu09.pdf}

  %%%%%%%%%%%%%
  %% POSTERS %%
  %%%%%%%%%%%%%
  % The user's posters are displayed here.
  %
  \section{Posters}

  \posterEntry
    {\textbf{Kevin Jalbert}, Cody-James LeBlanc, Christopher Forbes, Ramiro Liscano and Jeremy S. Bradbury}
    {Eclipticon: Eclipse Plugin for Concurrency Testing (\textbf{Best Poster Award})}
    {In 2011 Fall Meeting of the Consortium for Software Engineering Research (CSER)}
    {Markham, Ontario, Canada}
    {Nov. 2011}
    {}

  \posterEntry
    {\textbf{Kevin Jalbert}, David Kelk and Jeremy S. Bradbury}
    {ARC: Automatic Repair of Java Concurrency Bugs}
    {In 2011 Fall Meeting of the Consortium for Software Engineering Research (CSER)}
    {Markham, Ontario, Canada}
    {Nov. 2011}
    {}

  \posterEntry
    {\textbf{Kevin Jalbert} and Jeremy S. Bradbury}
    {Predicting Difficulty of Detecting Bugs using Source Code Metrics}
    {In 2011 Summer Meeting of the Consortium for Software Engineering Research (CSER)}
    {Kingston, Ontario, Canada}
    {Jun. 2011}
    {http://kevinjalbert.com/public/files/posters/CSER11-summer-poster.pdf}

  \posterEntry
    {\textbf{Kevin Jalbert} and Jeremy S. Bradbury}
    {A Tool for Automatically Repairing Concurrency Bugs}
    {In Technology Showcase at the 20th Annual International Conference on Computer Science and Software Engineering (CASCON 2010)}
    {Markham, Ontario, Canada}
    {Nov. 2010}
    {http://kevinjalbert.com/public/files/posters/CASCON10-poster.pdf}

  \posterEntry
    {\textbf{Kevin Jalbert} and Jeremy S. Bradbury}
    {A Tool for Automatically Repairing Concurrency Bugs}
    {In 2010 Fall Meeting of the Consortium for Software Engineering Research (CSER)}
    {Markham, Ontario, Canada}
    {Nov. 2010}
    {http://kevinjalbert.com/public/files/posters/CSER10-fall-poster.pdf}

  \posterEntry
    {\textbf{Kevin Jalbert} and Jeremy S. Bradbury}
    {Using Bug Patterns in the Regression Testing of Concurrent Software}
    {In Technology Showcase at the 19th Annual International Conference on Computer Science and Software Engineering (CASCON 2009)}
    {Markham, Ontario, Canada}
    {Nov. 2009}
    {http://kevinjalbert.com/public/files/posters/CASCON09-poster.pdf}

  \posterEntry
    {\textbf{Kevin Jalbert} and Jeremy S. Bradbury}
    {Using Bug Patterns in the Regression Testing of Concurrent Software}
    {In 2009 Fall Meeting of the Consortium for Software Engineering Research (CSER)}
    {Markham, Ontario, Canada}
    {Nov. 2009}
    {http://kevinjalbert.com/public/files/publications/CSER09-fall-poster.pdf}

  %%%%%%%%%%%%%%%%%%%
  %% PRESENTATIONS %%
  %%%%%%%%%%%%%%%%%%%
  % The user's posters are displayed here.
  %
  \section{Presentations}

  \presentationEntry
    {ARC: Automatic Repair of Concurrency Bugs}
    {In 2011 Fall Meeting of the Consortium for Software Engineering Research (CSER)}
    {Markham, Ontario, Canada}
    {Nov. 2011}
    {Discussed a technique to automatically repair concurrency bugs in Java programs}
    {}

  \presentationEntry
    {Eclipticon}
    {Eclipse DemoCamps 2011/Toronto}
    {Toronto, Ontario, Canada}
    {Jun. 2011}
    {Detailed and demonstrated an Eclipse plugin for testing multi--threaded Java programs}
    {http://wiki.eclipse.org/Eclipse_DemoCamps_Indigo_2011/Toronto}

  %%%%%%%%%%%%%%%%%%%%%%%%%%%%%%%
  %% ORGANIZATION INVOLVEMENTS %%
  %%%%%%%%%%%%%%%%%%%%%%%%%%%%%%%
  % The user's organization involvements are displayed here.
  %
  \section{Organization Involvements}

  \organizationInvolvementEntry
    {2011 Fall Meeting of the Consortium for Software Engineering Research (CSER)}
    {Poster Session Co-organizer}
    {Oct. 2011}
    {Nov. 2011}
    {}

  \organizationInvolvementEntry
    {Symposium on Search Based Software Engineering (SSBSE 2011)}
    {Sub-Reviewer}
    {May. 2011}
    {Jun. 2011}
    {http://www.ssbse.org/2011/}

  %%%%%%%%%%%%%%%%%%%%%%%%%%%
  %% PROGRAMMING LANGUAGES %%
  %%%%%%%%%%%%%%%%%%%%%%%%%%%
  % The user's programming languages and comfort levels are displayed here.
  %
  \section{Programming Languages}

  % Simple version without the comfort graphs
  %\hide{
    \generalEntry{Bash, C\#, C++, Java, \LaTeX, Python, Ruby, Visual Basic}
  %}

  % Detailed version with skill level information
  \hide{
    \generalEntry{\scriptsize{The following represent the skill set I have and my personal comfort level of using that skill} \\}
    \skillEntry{Bash}{7}
    \skillEntry{C\#}{8}
    \skillEntry{C++}{6}
    \skillEntry{Java}{10}
    \skillEntry{\LaTeX}{9}
    \skillEntry{Python}{8}
    \skillEntry{Ruby}{7}
    \skillEntry{Visual Basic}{8}
  }

  %%%%%%%%%%%%%%%%%%%%%
  %% SOFTWARE SKILLS %%
  %%%%%%%%%%%%%%%%%%%%%
  % The user's software skills and comfort levels are displayed here.
  %
  \section{Software Skills}

  % Simple version without the comfort graphs
  %\hide{
    \generalEntry{Software Engineering Design, UML, Concurrency, Software Quality and Testing, JUnit, Linux, Git, Word Processing}
  %}

  % Detailed version with skill level information
  \hide{
    \generalEntry{\scriptsize{The following represent the skill set I have and my personal comfort level of using that skill} \\}
    \skillEntry{Software Engineering Design}{9}
    \skillEntry{UML}{9}
    \skillEntry{Concurrency}{8}
    \skillEntry{Software Quality and Testing}{7}
    \skillEntry{Linux}{10}
    \skillEntry{Git}{8}
    \skillEntry{Word Processing}{9}
  }

\end{document}
